\documentclass[12pt,a4paper]{article}
\usepackage[utf8]{inputenc}
\usepackage[english]{babel}
\usepackage{geometry}
\usepackage{fancyhdr}
\usepackage{graphicx}
\usepackage{float}
\usepackage{tabularx}
\usepackage{longtable}
\usepackage{array}
\usepackage{booktabs}
\usepackage{multirow}
\usepackage{listings}
\usepackage{xcolor}
\usepackage{tikz}
\usepackage{forest}
\usepackage{hyperref}
\usepackage{amsmath}
\usepackage{amsfonts}
\usepackage{amssymb}
\usepackage{enumitem}
\usepackage{caption}
\usepackage{subcaption}

% Page setup
\geometry{
    left=2.5cm,
    right=2.5cm,
    top=2.5cm,
    bottom=2.5cm
}

% Header and footer
\pagestyle{fancy}
\fancyhf{}
\rhead{HIV Clinic Management System}
\lhead{Software Design Specification}
\cfoot{\thepage}

% Code listing style
\lstset{
    basicstyle=\ttfamily\small,
    breaklines=true,
    frame=single,
    language=Java,
    numbers=left,
    numberstyle=\tiny,
    showstringspaces=false,
    tabsize=2,
    commentstyle=\color{gray},
    keywordstyle=\color{blue},
    stringstyle=\color{red}
}

% TikZ libraries
\usetikzlibrary{shapes.geometric, arrows, positioning, fit, backgrounds}

% Define colors
\definecolor{packagecolor}{RGB}{173, 216, 230}
\definecolor{classcolor}{RGB}{255, 218, 185}
\definecolor{interfacecolor}{RGB}{221, 160, 221}

% Title page
\title{
    \vspace{-2cm}
    \Huge\textbf{HIV Clinic Management System}\\
    \vspace{1cm}
    \Large\textbf{Software Design Specification}\\
    \vspace{2cm}
    \normalsize Version 1.0
}

\author{}
\date{
    \vspace{4cm}
    – Ho Chi Minh City, January 2025 –
}

\begin{document}

\maketitle
\thispagestyle{empty}

\newpage

% Record of changes
\section*{Record of Changes}
\begin{longtable}{|p{3cm}|p{2cm}|p{3cm}|p{6cm}|}
\hline
\textbf{Date} & \textbf{A*M, D} & \textbf{In charge} & \textbf{Change Description} \\
\hline
2025-01-07 & A & System Architect & Initial creation of SDS document \\
\hline
2025-01-07 & A & System Architect & Added comprehensive system architecture \\
\hline
2025-01-07 & A & System Architect & Added detailed class diagrams and specifications \\
\hline
2025-01-07 & A & System Architect & Added database design and API documentation \\
\hline
\end{longtable}

\textit{*A - Added M - Modified D - Deleted}

\newpage

\tableofcontents

\newpage

\section{Introduction}

\subsection{Purpose}
This Software Design Specification (SDS) document provides a comprehensive overview of the software design for the HIV Clinic Management System. It serves as a detailed guide for developers, architects, and stakeholders to understand the system's architecture, design patterns, and technical implementation details.

\subsection{Scope}
The HIV Clinic Management System is a web-based application designed to manage HIV clinic operations including patient management, appointment scheduling, ARV treatment tracking, and administrative functions. The system implements a multi-tier architecture with Spring Boot backend and React frontend.

\subsection{References}
\begin{itemize}
    \item Requirements and Design Specification (RDS) Document
    \item Software Requirements Specification (SRS) Document
    \item Database Schema Documentation
    \item API Documentation
\end{itemize}

\section{System Architecture}

\subsection{Overall System Architecture}

The HIV Clinic Management System follows a three-tier architecture pattern:

\begin{figure}[H]
\centering
\fbox{
\begin{minipage}{0.9\textwidth}
\centering
\vspace{3cm}
\textbf{HIV Clinic Management System - System Architecture}\\
\vspace{0.5cm}
\textit{Three-tier architecture showing Frontend (React), Backend (Spring Boot), and Database (MS SQL Server) layers with their interactions}\\
\vspace{0.5cm}
\textit{Ref: system\_architecture.mermaid}\\
\vspace{3cm}
\end{minipage}
}
\caption{System Architecture Overview}
\label{fig:system-architecture}
\end{figure}

\subsubsection{Presentation Layer}
\begin{itemize}
    \item \textbf{Technology:} React 18.2.0 with Vite
    \item \textbf{Routing:} React Router DOM 6.8.0
    \item \textbf{HTTP Client:} Axios 1.6.0
    \item \textbf{State Management:} React Context API
    \item \textbf{Testing:} Vitest with React Testing Library
\end{itemize}

\subsubsection{Business Logic Layer}
\begin{itemize}
    \item \textbf{Framework:} Spring Boot 3.2.0
    \item \textbf{Language:} Java 17
    \item \textbf{Security:} Spring Security with JWT
    \item \textbf{Data Access:} Spring Data JPA with Hibernate
    \item \textbf{Build Tool:} Maven
\end{itemize}

\subsubsection{Data Layer}
\begin{itemize}
    \item \textbf{Database:} Microsoft SQL Server
    \item \textbf{ORM:} Hibernate 6.x
    \item \textbf{Connection Pool:} HikariCP
    \item \textbf{Migration:} JPA DDL Auto-update
\end{itemize}

\subsection{Component Architecture}

\begin{figure}[H]
\centering
\fbox{
\begin{minipage}{0.9\textwidth}
\centering
\vspace{3cm}
\textbf{HIV Clinic Management System - Component Diagram}\\
\vspace{0.5cm}
\textit{Detailed component diagram showing all system modules and their dependencies}\\
\vspace{0.5cm}
\textit{Ref: component\_diagram.plantuml}\\
\vspace{3cm}
\end{minipage}
}
\caption{Component Architecture}
\label{fig:component-diagram}
\end{figure}

\section{Backend Design}

\subsection{Package Structure}

The backend follows a layered architecture pattern with the following package structure:

\begin{lstlisting}[language=text, caption=Backend Package Structure]
com.hivclinic
├── HivClinicBackendApplication.java
├── config/
│   ├── CorsConfig.java
│   ├── SecurityConfig.java
│   └── JwtConfig.java
├── controller/
│   ├── AuthController.java
│   ├── PatientController.java
│   ├── AppointmentController.java
│   ├── ARVController.java
│   └── NotificationController.java
├── dto/
│   ├── request/
│   └── response/
├── exception/
│   ├── GlobalExceptionHandler.java
│   └── CustomExceptions.java
├── mapper/
│   ├── PatientMapper.java
│   └── AppointmentMapper.java
├── model/
│   ├── User.java
│   ├── Patient.java
│   ├── Appointment.java
│   ├── ARVTreatment.java
│   └── Notification.java
├── repository/
│   ├── UserRepository.java
│   ├── PatientRepository.java
│   ├── AppointmentRepository.java
│   ├── ARVRepository.java
│   └── NotificationRepository.java
├── service/
│   ├── UserService.java
│   ├── PatientService.java
│   ├── AppointmentService.java
│   ├── ARVService.java
│   └── NotificationService.java
└── validation/
    ├── UserValidator.java
    └── AppointmentValidator.java
\end{lstlisting}

\subsection{Database Design}

\begin{figure}[H]
\centering
\fbox{
\begin{minipage}{0.9\textwidth}
\centering
\vspace{3cm}
\textbf{HIV Clinic Management System - Database Schema ERD}\\
\vspace{0.5cm}
\textit{Entity Relationship Diagram showing all database tables and their relationships}\\
\vspace{0.5cm}
\textit{Ref: database\_schema\_erd.mermaid}\\
\vspace{3cm}
\end{minipage}
}
\caption{Database Schema ERD}
\label{fig:database-schema}
\end{figure}

\subsubsection{Core Entities}

\paragraph{User Entity}
\begin{lstlisting}[language=Java, caption=User Entity]
@Entity
@Table(name = "users")
public class User {
    @Id
    @GeneratedValue(strategy = GenerationType.IDENTITY)
    private Long id;
    
    @Column(unique = true, nullable = false)
    private String username;
    
    @Column(nullable = false)
    private String password;
    
    @Enumerated(EnumType.STRING)
    private Role role;
    
    private String email;
    private String fullName;
    private LocalDateTime createdAt;
    private LocalDateTime updatedAt;
    private Boolean active;
}
\end{lstlisting}

\paragraph{Patient Entity}
\begin{lstlisting}[language=Java, caption=Patient Entity]
@Entity
@Table(name = "patients")
public class Patient {
    @Id
    @GeneratedValue(strategy = GenerationType.IDENTITY)
    private Long id;
    
    @OneToOne
    @JoinColumn(name = "user_id")
    private User user;
    
    private String patientCode;
    private String nationalId;
    private String phoneNumber;
    private String address;
    private LocalDate dateOfBirth;
    private String gender;
    private LocalDate diagnosisDate;
    private String hivStatus;
    private String emergencyContact;
    private String medicalHistory;
}
\end{lstlisting}

\subsection{Security Architecture}

\begin{figure}[H]
\centering
\fbox{
\begin{minipage}{0.9\textwidth}
\centering
\vspace{3cm}
\textbf{HIV Clinic Management System - Authentication Flow}\\
\vspace{0.5cm}
\textit{Authentication sequence diagram showing JWT token-based authentication process}\\
\vspace{0.5cm}
\textit{Ref: authentication\_sequence.mermaid}\\
\vspace{3cm}
\end{minipage}
}
\caption{Authentication Flow}
\label{fig:authentication-flow}
\end{figure}

\subsubsection{JWT Configuration}
\begin{lstlisting}[language=Java, caption=JWT Configuration]
@Configuration
public class JwtConfig {
    @Value("${app.jwt.secret}")
    private String jwtSecret;
    
    @Value("${app.jwt.expiration-ms}")
    private int jwtExpirationMs;
    
    public String generateJwtToken(Authentication authentication) {
        return Jwts.builder()
                .setSubject(authentication.getName())
                .setIssuedAt(new Date())
                .setExpiration(new Date(System.currentTimeMillis() + jwtExpirationMs))
                .signWith(SignatureAlgorithm.HS512, jwtSecret)
                .compact();
    }
}
\end{lstlisting}

\section{Frontend Design}

\subsection{Component Architecture}

The frontend follows a feature-based architecture with the following structure:

\begin{lstlisting}[language=text, caption=Frontend Component Structure]
src/
├── main.jsx
├── App.jsx
├── index.css
├── App.css
├── components/
│   ├── ui/
│   │   ├── Button.jsx
│   │   ├── Modal.jsx
│   │   └── Table.jsx
│   ├── layout/
│   │   ├── Header.jsx
│   │   ├── Sidebar.jsx
│   │   └── Footer.jsx
│   ├── notifications/
│   │   ├── NotificationList.jsx
│   │   └── NotificationItem.jsx
│   ├── arv/
│   │   ├── ARVList.jsx
│   │   └── ARVForm.jsx
│   ├── schedule/
│   │   ├── Calendar.jsx
│   │   └── AppointmentForm.jsx
│   └── manager/
│       ├── UserManagement.jsx
│       └── SystemSettings.jsx
├── features/
│   ├── auth/
│   │   ├── Login.jsx
│   │   ├── Register.jsx
│   │   └── AuthContext.jsx
│   ├── Admin/
│   │   ├── Dashboard.jsx
│   │   └── UserManagement.jsx
│   ├── Doctor/
│   │   ├── PatientList.jsx
│   │   └── MedicalRecords.jsx
│   ├── Manager/
│   │   ├── Reports.jsx
│   │   └── Analytics.jsx
│   └── Customer/
│       ├── Profile.jsx
│       └── Appointments.jsx
├── contexts/
│   ├── AuthContext.jsx
│   └── ThemeContext.jsx
├── services/
│   ├── api.js
│   ├── authService.js
│   └── patientService.js
└── utils/
    ├── constants.js
    └── helpers.js
\end{lstlisting}

\subsection{State Management}

\begin{figure}[H]
\centering
\fbox{
\begin{minipage}{0.9\textwidth}
\centering
\vspace{3cm}
\textbf{HIV Clinic Management System - Data Flow Diagram}\\
\vspace{0.5cm}
\textit{Data flow diagram showing how data moves through the React application}\\
\vspace{0.5cm}
\textit{Ref: data\_flow\_diagram.mermaid}\\
\vspace{3cm}
\end{minipage}
}
\caption{Frontend Data Flow}
\label{fig:data-flow}
\end{figure}

\subsubsection{Authentication Context}
\begin{lstlisting}[language=JavaScript, caption=Authentication Context]
import { createContext, useContext, useState, useEffect } from 'react';
import { authService } from '../services/authService';

const AuthContext = createContext();

export const useAuth = () => {
    const context = useContext(AuthContext);
    if (!context) {
        throw new Error('useAuth must be used within AuthProvider');
    }
    return context;
};

export const AuthProvider = ({ children }) => {
    const [user, setUser] = useState(null);
    const [loading, setLoading] = useState(true);

    useEffect(() => {
        const initializeAuth = async () => {
            const token = localStorage.getItem('token');
            if (token) {
                try {
                    const userData = await authService.validateToken(token);
                    setUser(userData);
                } catch (error) {
                    localStorage.removeItem('token');
                }
            }
            setLoading(false);
        };
        initializeAuth();
    }, []);

    const login = async (credentials) => {
        const response = await authService.login(credentials);
        localStorage.setItem('token', response.token);
        setUser(response.user);
        return response;
    };

    const logout = () => {
        localStorage.removeItem('token');
        setUser(null);
    };

    const value = {
        user,
        login,
        logout,
        loading
    };

    return (
        <AuthContext.Provider value={value}>
            {children}
        </AuthContext.Provider>
    );
};
\end{lstlisting}

\section{API Design}

\subsection{RESTful API Structure}

The system implements a RESTful API with the following endpoints:

\begin{longtable}{|p{1.5cm}|p{4cm}|p{8cm}|}
\hline
\textbf{Method} & \textbf{Endpoint} & \textbf{Description} \\
\hline
POST & /api/auth/login & User authentication \\
\hline
POST & /api/auth/register & User registration \\
\hline
GET & /api/auth/me & Get current user info \\
\hline
GET & /api/patients & Get all patients \\
\hline
POST & /api/patients & Create new patient \\
\hline
PUT & /api/patients/\{id\} & Update patient \\
\hline
DELETE & /api/patients/\{id\} & Delete patient \\
\hline
GET & /api/appointments & Get appointments \\
\hline
POST & /api/appointments & Create appointment \\
\hline
PUT & /api/appointments/\{id\} & Update appointment \\
\hline
DELETE & /api/appointments/\{id\} & Cancel appointment \\
\hline
GET & /api/arv-treatments & Get ARV treatments \\
\hline
POST & /api/arv-treatments & Create ARV treatment \\
\hline
PUT & /api/arv-treatments/\{id\} & Update ARV treatment \\
\hline
GET & /api/notifications & Get notifications \\
\hline
POST & /api/notifications & Create notification \\
\hline
PUT & /api/notifications/\{id\}/read & Mark as read \\
\hline
\end{longtable}

\subsection{Request/Response Patterns}

\subsubsection{Authentication Request}
\begin{lstlisting}[language=JSON, caption=Login Request]
{
    "username": "patient123",
    "password": "securePassword"
}
\end{lstlisting}

\subsubsection{Authentication Response}
\begin{lstlisting}[language=JSON, caption=Login Response]
{
    "token": "eyJhbGciOiJIUzUxMiJ9...",
    "user": {
        "id": 1,
        "username": "patient123",
        "role": "PATIENT",
        "fullName": "John Doe",
        "email": "john.doe@example.com"
    }
}
\end{lstlisting}

\section{Security Design}

\subsection{Authentication and Authorization}

\begin{figure}[H]
\centering
\fbox{
\begin{minipage}{0.9\textwidth}
\centering
\vspace{3cm}
\textbf{HIV Clinic Management System - Security Architecture}\\
\vspace{0.5cm}
\textit{Security architecture diagram showing authentication, authorization, and data protection layers}\\
\vspace{0.5cm}
\textit{Ref: authentication\_class\_diagram.plantuml}\\
\vspace{3cm}
\end{minipage}
}
\caption{Security Architecture}
\label{fig:security-architecture}
\end{figure}

\subsection{Role-Based Access Control}

\begin{longtable}{|p{2cm}|p{3cm}|p{9cm}|}
\hline
\textbf{Role} & \textbf{Level} & \textbf{Permissions} \\
\hline
ADMIN & Full Access & Complete system access, user management, system configuration \\
\hline
MANAGER & Management & User oversight, reports, clinic management, limited system settings \\
\hline
DOCTOR & Clinical & Patient records, appointments, ARV treatments, medical data \\
\hline
PATIENT & User & Personal profile, appointments, medical records (own), notifications \\
\hline
\end{longtable}

\section{Performance and Scalability}

\subsection{Performance Considerations}

\begin{itemize}
    \item \textbf{Database Optimization:} Connection pooling with HikariCP
    \item \textbf{Caching Strategy:} Application-level caching for frequently accessed data
    \item \textbf{Lazy Loading:} JPA lazy loading for related entities
    \item \textbf{Frontend Optimization:} Code splitting and lazy loading of React components
    \item \textbf{API Pagination:} Paginated responses for large datasets
\end{itemize}

\subsection{Scalability Architecture}

\begin{figure}[H]
\centering
\fbox{
\begin{minipage}{0.9\textwidth}
\centering
\vspace{3cm}
\textbf{HIV Clinic Management System - Deployment Architecture}\\
\vspace{0.5cm}
\textit{Deployment diagram showing production environment setup with load balancing and database clustering}\\
\vspace{0.5cm}
\textit{Ref: deployment\_diagram.plantuml}\\
\vspace{3cm}
\end{minipage}
}
\caption{Deployment Architecture}
\label{fig:deployment-architecture}
\end{figure}

\section{Testing Strategy}

\subsection{Backend Testing}

\begin{itemize}
    \item \textbf{Unit Testing:} JUnit 5 for service layer testing
    \item \textbf{Integration Testing:} Spring Boot Test for controller testing
    \item \textbf{Repository Testing:} @DataJpaTest for repository layer
    \item \textbf{Security Testing:} Spring Security Test for authentication
\end{itemize}

\subsection{Frontend Testing}

\begin{itemize}
    \item \textbf{Unit Testing:} Vitest for component testing
    \item \textbf{Integration Testing:} React Testing Library for user interactions
    \item \textbf{E2E Testing:} Planned Cypress integration
    \item \textbf{Coverage:} Minimum 80\% code coverage requirement
\end{itemize}

\section{Deployment and Configuration}

\subsection{Environment Configuration}

\begin{lstlisting}[language=properties, caption=Production Configuration]
# Production Database Configuration
spring.datasource.url=jdbc:sqlserver://prod-db-server:1433;databaseName=hiv_clinic_prod
spring.datasource.username=${DB_USERNAME}
spring.datasource.password=${DB_PASSWORD}

# Security Configuration
app.jwt.secret=${JWT_SECRET}
app.jwt.expiration-ms=3600000

# CORS Configuration
app.cors.allowed-origins=${FRONTEND_URL}

# Logging Configuration
logging.level.com.hivclinic=INFO
logging.level.org.springframework.security=WARN
\end{lstlisting}

\subsection{Build and Deployment}

\begin{itemize}
    \item \textbf{Backend Build:} Maven for Java application packaging
    \item \textbf{Frontend Build:} Vite for React application bundling
    \item \textbf{Containerization:} Docker containers for deployment
    \item \textbf{CI/CD:} GitHub Actions for automated testing and deployment
\end{itemize}

\section{Conclusion}

The HIV Clinic Management System implements a robust, scalable architecture using modern technologies and best practices. The system provides comprehensive functionality for managing HIV clinic operations while maintaining security, performance, and usability standards.

The modular design allows for easy maintenance and future enhancements, while the comprehensive testing strategy ensures system reliability and quality.

\end{document}