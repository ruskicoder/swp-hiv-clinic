\documentclass[12pt,a4paper]{article}
\usepackage[utf8]{inputenc}
\usepackage[english]{babel}
\usepackage{geometry}
\usepackage{fancyhdr}
\usepackage{graphicx}
\usepackage{longtable}
\usepackage{array}
\usepackage{booktabs}
\usepackage{xcolor}
\usepackage{hyperref}
\usepackage{listings}
\usepackage{enumitem}
\usepackage{float}
\usepackage{amsmath}
\usepackage{amsfonts}
\usepackage{amssymb}

\geometry{margin=1in}
\pagestyle{fancy}
\fancyhf{}
\rhead{\thepage}
\lhead{HIV Clinic Final Release}

\title{\textbf{Final Release Document\\HIV Clinic Management System}}
\author{Version: 1.0}
\date{January 2025}

\begin{document}

\maketitle
\thispagestyle{empty}

\newpage

\section*{Record of Changes}

\begin{longtable}{|p{2cm}|p{2cm}|p{1cm}|p{3cm}|p{6cm}|}
\hline
\textbf{Version} & \textbf{Date} & \textbf{A*M, D} & \textbf{In charge} & \textbf{Change Description} \\
\hline
V1.0 & 07/01/2025 & A & Development Team & Initial HIV Clinic System Final Release document based on implemented codebase \\
\hline
\end{longtable}

\textit{*A - Added M - Modified D - Deleted}

\newpage

\tableofcontents

\newpage

\section{Executive Summary}

The HIV Clinic Management System is a comprehensive web-based application designed to streamline clinic operations, enhance patient care, and improve treatment management for HIV patients. This final release document provides a complete overview of the system's features, installation procedures, user guides, and technical specifications.

\subsection{Project Overview}

The system has been successfully developed and tested, providing the following key capabilities:

\begin{itemize}
    \item \textbf{User Management}: Role-based access control with secure authentication
    \item \textbf{Appointment Management}: Comprehensive scheduling and booking system
    \item \textbf{ARV Treatment Tracking}: HIV medication management and adherence monitoring
    \item \textbf{Patient Records}: Secure medical record management
    \item \textbf{Notification System}: Automated reminders for appointments and medications
    \item \textbf{Administrative Functions}: User management and system analytics
\end{itemize}

\subsection{Technology Stack}

\begin{longtable}{|p{3cm}|p{3cm}|p{8cm}|}
\hline
\textbf{Component} & \textbf{Technology} & \textbf{Version} \\
\hline
Backend Framework & Spring Boot & 3.2.0 \\
\hline
Programming Language & Java & 17 \\
\hline
Database & Microsoft SQL Server & 2019+ \\
\hline
Frontend Framework & React & 18.2.0 \\
\hline
Build Tool & Maven & 3.9.6 \\
\hline
Authentication & JWT & 0.11.5 \\
\hline
\end{longtable}

\section{System Features}

\subsection{Core Features}

\subsubsection{Authentication and User Management}

The system implements secure JWT-based authentication with role-based access control:

\begin{itemize}
    \item \textbf{User Registration}: New users can register with role selection
    \item \textbf{Secure Login}: JWT token-based authentication
    \item \textbf{Role-based Access}: Patient, Doctor, Admin, and Manager roles
    \item \textbf{Profile Management}: Users can update personal information
    \item \textbf{Session Management}: Secure session handling with timeout
\end{itemize}

\subsubsection{Appointment Management}

Comprehensive appointment scheduling system with the following features:

\begin{itemize}
    \item \textbf{Appointment Booking}: Patients can book appointments with available doctors
    \item \textbf{Availability Management}: Doctors can manage their availability slots
    \item \textbf{Appointment Status}: Track scheduled, confirmed, cancelled, and completed appointments
    \item \textbf{Cancellation Handling}: Support for appointment cancellations with reasons
    \item \textbf{Reminder System}: Automated appointment reminders
\end{itemize}

\subsubsection{ARV Treatment Management}

Specialized HIV treatment tracking system:

\begin{itemize}
    \item \textbf{Treatment Prescription}: Doctors can prescribe ARV medications
    \item \textbf{Medication Tracking}: Monitor medication adherence and side effects
    \item \textbf{Dosage Management}: Track medication dosage and frequency
    \item \textbf{Treatment History}: Comprehensive treatment history and notes
    \item \textbf{Medication Reminders}: Automated medication reminder system
\end{itemize}

\subsubsection{Patient Records}

Secure medical record management:

\begin{itemize}
    \item \textbf{Medical History}: Comprehensive patient medical history
    \item \textbf{Allergy Tracking}: Patient allergy information
    \item \textbf{Current Medications}: Active medication tracking
    \item \textbf{Emergency Contacts}: Emergency contact information
    \item \textbf{Privacy Controls}: Patient privacy settings and data protection
\end{itemize}

\subsubsection{Notification System}

Automated notification and reminder system:

\begin{itemize}
    \item \textbf{Appointment Reminders}: 24-hour, 1-hour, and 30-minute reminders
    \item \textbf{Medication Reminders}: Daily medication adherence reminders
    \item \textbf{System Notifications}: General system notifications
    \item \textbf{Template Management}: Configurable notification templates
    \item \textbf{Delivery Tracking}: Notification delivery status tracking
\end{itemize}

\subsubsection{Administrative Functions}

Comprehensive administrative capabilities:

\begin{itemize}
    \item \textbf{User Management}: Admin can manage user accounts and roles
    \item \textbf{System Analytics}: Managers can view system statistics and reports
    \item \textbf{Health Monitoring}: System health checks and monitoring
    \item \textbf{Audit Logging}: Comprehensive activity logging
    \item \textbf{System Settings}: Configurable system parameters
\end{itemize}

\section{Installation Guide}

\subsection{Prerequisites}

Before installing the HIV Clinic Management System, ensure the following prerequisites are met:

\begin{longtable}{|p{3cm}|p{3cm}|p{8cm}|}
\hline
\textbf{Requirement} & \textbf{Version} & \textbf{Description} \\
\hline
Java & 17 or higher & Required for Spring Boot backend \\
\hline
Node.js & 18 or higher & Required for React frontend \\
\hline
SQL Server & 2019 or higher & Database server \\
\hline
Maven & 3.9+ & Build tool for backend \\
\hline
npm & Latest & Package manager for frontend \\
\hline
Git & Latest & Version control system \\
\hline
\end{longtable}

\subsection{Database Setup}

\subsubsection{Step 1: Create Database}

\begin{lstlisting}[language=SQL, caption=Database Creation]
-- Connect to SQL Server and create database
CREATE DATABASE hiv_clinic;
GO

-- Use the database
USE hiv_clinic;
GO
\end{lstlisting}

\subsubsection{Step 2: Execute Schema Script}

Run the database schema creation script:

\begin{lstlisting}[language=bash, caption=Schema Setup]
# Navigate to database scripts directory
cd src/main/resources/db/

# Execute schema creation
sqlcmd -S localhost -d hiv_clinic -i schema.sql
\end{lstlisting}

\subsubsection{Step 3: Populate Initial Data}

Execute the data population script:

\begin{lstlisting}[language=bash, caption=Data Population]
# Execute data population
sqlcmd -S localhost -d hiv_clinic -i data.sql
\end{lstlisting}

\subsection{Backend Installation}

\subsubsection{Step 1: Clone Repository}

\begin{lstlisting}[language=bash, caption=Repository Clone]
git clone https://github.com/ruskicoder/swp-hiv-clinic.git
cd swp-hiv-clinic
\end{lstlisting}

\subsubsection{Step 2: Configure Database Connection}

Update the database configuration in `src/main/resources/application.properties`:

\begin{lstlisting}[language=properties, caption=Database Configuration]
# Database Configuration
spring.datasource.url=jdbc:sqlserver://localhost:1433;databaseName=hiv_clinic;encrypt=true;trustServerCertificate=true
spring.datasource.username=your_username
spring.datasource.password=your_password
spring.datasource.driver-class-name=com.microsoft.sqlserver.jdbc.SQLServerDriver

# JPA Configuration
spring.jpa.hibernate.ddl-auto=update
spring.jpa.show-sql=true
spring.jpa.properties.hibernate.dialect=org.hibernate.dialect.SQLServerDialect

# JWT Configuration
jwt.secret=your_jwt_secret_key_here
jwt.expiration=86400000
\end{lstlisting}

\subsubsection{Step 3: Build and Run Backend}

\begin{lstlisting}[language=bash, caption=Backend Build and Run]
# Install dependencies
mvn clean install

# Run the application
mvn spring-boot:run
\end{lstlisting}

The backend will start on `http://localhost:8080`

\subsection{Frontend Installation}

\subsubsection{Step 1: Install Dependencies}

\begin{lstlisting}[language=bash, caption=Frontend Dependencies]
# Install Node.js dependencies
npm install
\end{lstlisting}

\subsubsection{Step 2: Configure Environment}

Create a `.env` file in the root directory:

\begin{lstlisting}[language=properties, caption=Environment Configuration]
VITE_API_BASE_URL=http://localhost:8080/api
VITE_APP_NAME=HIV Clinic Management System
VITE_ENABLE_MOCK_DATA=false
\end{lstlisting}

\subsubsection{Step 3: Run Frontend}

\begin{lstlisting}[language=bash, caption=Frontend Run]
# Start development server
npm run dev
\end{lstlisting}

The frontend will start on `http://localhost:3000`

\subsection{Automated Setup}

For Windows users, use the automated setup script:

\begin{lstlisting}[language=bash, caption=Automated Setup]
# Run automated setup script
./setup-database.bat
\end{lstlisting}

For Unix/Linux/Mac users:

\begin{lstlisting}[language=bash, caption=Automated Setup]
# Run automated setup script
./setup-database.sh
\end{lstlisting}

\section{User Guide}

\subsection{Getting Started}

\subsubsection{First Time Setup}

1. \textbf{Access the System}: Open your web browser and navigate to `http://localhost:3000`

2. \textbf{Register Account}: Click "Register" to create a new account
   - Choose your role (Patient, Doctor, Admin, Manager)
   - Fill in required information
   - Set a strong password

3. \textbf{Login}: Use your credentials to log into the system

4. \textbf{Complete Profile}: Update your profile information as needed

\subsubsection{System Navigation}

The system provides role-based navigation:

\begin{itemize}
    \item \textbf{Patient Dashboard}: Appointment booking, medical records, medication reminders
    \item \textbf{Doctor Dashboard}: Patient management, appointment scheduling, treatment prescriptions
    \item \textbf{Admin Dashboard}: User management, system settings, oversight functions
    \item \textbf{Manager Dashboard}: Analytics, reporting, operational insights
\end{itemize}

\subsection{Patient User Guide}

\subsubsection{Booking Appointments}

1. \textbf{Access Appointment Booking}:
   - Navigate to "Book Appointment" from the dashboard
   - Select your preferred doctor
   - Choose available time slot
   - Provide appointment reason (optional)
   - Confirm booking

2. \textbf{Managing Appointments}:
   - View upcoming appointments
   - Cancel appointments if needed
   - Receive automated reminders

\subsubsection{Managing Medical Records}

1. \textbf{View Records}:
   - Access "My Medical Records" from dashboard
   - View treatment history
   - Check current medications

2. \textbf{Privacy Settings}:
   - Configure privacy preferences
   - Control data visibility
   - Manage notification settings

\subsubsection{Medication Management}

1. \textbf{View Medications}:
   - Check current ARV treatments
   - View medication schedules
   - Track adherence

2. \textbf{Receive Reminders}:
   - Get daily medication reminders
   - Track medication compliance
   - Report side effects

\subsection{Doctor User Guide}

\subsubsection{Managing Availability}

1. \textbf{Set Availability}:
   - Navigate to "Availability Management"
   - Set available time slots
   - Configure appointment duration
   - Update availability as needed

2. \textbf{View Schedule}:
   - Check daily appointment schedule
   - View patient information
   - Prepare for consultations

\subsubsection{Patient Management}

1. \textbf{Access Patient Records}:
   - Search for patients
   - View medical history
   - Update patient information

2. \textbf{Prescribe Treatments}:
   - Create ARV treatment plans
   - Set medication schedules
   - Monitor treatment progress

\subsubsection{Appointment Management}

1. \textbf{View Appointments}:
   - Check daily schedule
   - Review patient information
   - Manage appointment status

2. \textbf{Update Records}:
   - Document consultation notes
   - Update treatment plans
   - Record patient progress

\subsection{Admin User Guide}

\subsubsection{User Management}

1. \textbf{Manage Users}:
   - View all system users
   - Create new user accounts
   - Update user roles and permissions
   - Activate/deactivate accounts

2. \textbf{System Oversight}:
   - Monitor system usage
   - Review audit logs
   - Manage system settings

\subsubsection{System Configuration}

1. \textbf{Configure Settings}:
   - Set system parameters
   - Configure notification templates
   - Manage security settings

2. \textbf{Monitor Health}:
   - Check system status
   - Review performance metrics
   - Monitor database health

\subsection{Manager User Guide}

\subsubsection{Analytics and Reporting}

1. \textbf{Generate Reports}:
   - Appointment statistics
   - Treatment compliance rates
   - User activity reports
   - System performance metrics

2. \textbf{Operational Insights}:
   - Clinic efficiency analysis
   - Resource utilization
   - Patient satisfaction metrics

\section{API Documentation}

\subsection{Authentication Endpoints}

\subsubsection{User Login}

\begin{lstlisting}[language=HTTP, caption=Login Request]
POST /api/auth/login
Content-Type: application/json

{
  "username": "user@example.com",
  "password": "password123"
}
\end{lstlisting}

\begin{lstlisting}[language=JSON, caption=Login Response]
{
  "success": true,
  "message": "Login successful",
  "data": {
    "token": "eyJhbGciOiJIUzI1NiIsInR5cCI6IkpXVCJ9...",
    "user": {
      "id": 1,
      "username": "user@example.com",
      "role": "PATIENT",
      "firstName": "John",
      "lastName": "Doe"
    }
  }
}
\end{lstlisting}

\subsubsection{User Registration}

\begin{lstlisting}[language=HTTP, caption=Registration Request]
POST /api/auth/register
Content-Type: application/json

{
  "username": "newuser@example.com",
  "email": "newuser@example.com",
  "password": "password123",
  "confirmPassword": "password123",
  "role": "PATIENT",
  "firstName": "Jane",
  "lastName": "Smith"
}
\end{lstlisting}

\subsection{Appointment Endpoints}

\subsubsection{Book Appointment}

\begin{lstlisting}[language=HTTP, caption=Book Appointment Request]
POST /api/appointments/book
Authorization: Bearer <token>
Content-Type: application/json

{
  "doctorId": 2,
  "appointmentDateTime": "2025-01-15T10:00:00",
  "reason": "Regular checkup"
}
\end{lstlisting}

\subsubsection{Get Patient Appointments}

\begin{lstlisting}[language=HTTP, caption=Get Appointments Request]
GET /api/appointments/patient/my-appointments
Authorization: Bearer <token>
\end{lstlisting}

\subsection{ARV Treatment Endpoints}

\subsubsection{Add ARV Treatment}

\begin{lstlisting}[language=HTTP, caption=Add Treatment Request]
POST /api/arv-treatments/add
Authorization: Bearer <token>
Content-Type: application/json

{
  "patientId": 1,
  "medicationName": "Tenofovir",
  "dosage": "300mg",
  "frequency": "Once daily",
  "startDate": "2025-01-07",
  "sideEffects": "Mild nausea",
  "doctorNotes": "Monitor kidney function"
}
\end{lstlisting}

\subsection{Notification Endpoints}

\subsubsection{Get Notifications}

\begin{lstlisting}[language=HTTP, caption=Get Notifications Request]
GET /api/notifications
Authorization: Bearer <token>
\end{lstlisting}

\section{Configuration Guide}

\subsection{Environment Variables}

\subsubsection{Backend Configuration}

\begin{lstlisting}[language=properties, caption=Backend Environment Variables]
# Database Configuration
SPRING_DATASOURCE_URL=jdbc:sqlserver://localhost:1433;databaseName=hiv_clinic
SPRING_DATASOURCE_USERNAME=your_username
SPRING_DATASOURCE_PASSWORD=your_password

# JWT Configuration
JWT_SECRET=your_jwt_secret_key_here
JWT_EXPIRATION=86400000

# Server Configuration
SERVER_PORT=8080
SERVER_CONTEXT_PATH=/api

# Logging Configuration
LOGGING_LEVEL_ROOT=INFO
LOGGING_LEVEL_COM_HIVCLINIC=DEBUG
\end{lstlisting}

\subsubsection{Frontend Configuration}

\begin{lstlisting}[language=properties, caption=Frontend Environment Variables]
# API Configuration
VITE_API_BASE_URL=http://localhost:8080/api
VITE_APP_NAME=HIV Clinic Management System

# Feature Flags
VITE_ENABLE_MOCK_DATA=false
VITE_ENABLE_DEBUG_MODE=false

# Build Configuration
VITE_BUILD_VERSION=1.0.0
VITE_BUILD_DATE=2025-01-07
\end{lstlisting}

\subsection{Database Configuration}

\subsubsection{Connection Pool Settings}

\begin{lstlisting}[language=properties, caption=Database Connection Pool]
# HikariCP Configuration
spring.datasource.hikari.maximum-pool-size=20
spring.datasource.hikari.minimum-idle=5
spring.datasource.hikari.idle-timeout=300000
spring.datasource.hikari.connection-timeout=20000
spring.datasource.hikari.max-lifetime=1200000
\end{lstlisting}

\subsubsection{JPA Configuration}

\begin{lstlisting}[language=properties, caption=JPA Configuration]
# JPA/Hibernate Configuration
spring.jpa.hibernate.ddl-auto=update
spring.jpa.show-sql=true
spring.jpa.properties.hibernate.dialect=org.hibernate.dialect.SQLServerDialect
spring.jpa.properties.hibernate.format_sql=true
spring.jpa.properties.hibernate.use_sql_comments=true
\end{lstlisting}

\section{Deployment Guide}

\subsection{Production Deployment}

\subsubsection{Backend Deployment}

1. \textbf{Build Application}:
\begin{lstlisting}[language=bash, caption=Backend Build]
mvn clean package -DskipTests
\end{lstlisting}

2. \textbf{Create Production Configuration}:
\begin{lstlisting}[language=properties, caption=Production Properties]
# Production Database
spring.datasource.url=jdbc:sqlserver://prod-server:1433;databaseName=hiv_clinic_prod
spring.datasource.username=${DB_USERNAME}
spring.datasource.password=${DB_PASSWORD}

# Production JWT
jwt.secret=${JWT_SECRET}
jwt.expiration=86400000

# Production Logging
logging.level.root=WARN
logging.level.com.hivclinic=INFO
\end{lstlisting}

3. \textbf{Deploy to Server}:
\begin{lstlisting}[language=bash, caption=Deployment]
# Copy JAR file to server
scp target/hiv-clinic-backend-*.jar user@server:/opt/hiv-clinic/

# Start application
java -jar hiv-clinic-backend-*.jar --spring.profiles.active=prod
\end{lstlisting}

\subsubsection{Frontend Deployment}

1. \textbf{Build for Production}:
\begin{lstlisting}[language=bash, caption=Frontend Build]
npm run build
\end{lstlisting}

2. \textbf{Deploy to Web Server}:
\begin{lstlisting}[language=bash, caption=Frontend Deployment]
# Copy build files to web server
scp -r dist/* user@server:/var/www/hiv-clinic/

# Configure web server (nginx example)
server {
    listen 80;
    server_name hiv-clinic.example.com;
    root /var/www/hiv-clinic;
    index index.html;
    
    location / {
        try_files $uri $uri/ /index.html;
    }
    
    location /api {
        proxy_pass http://localhost:8080;
        proxy_set_header Host $host;
        proxy_set_header X-Real-IP $remote_addr;
    }
}
\end{lstlisting}

\subsection{Docker Deployment}

\subsubsection{Backend Dockerfile}

\begin{lstlisting}[language=Dockerfile, caption=Backend Dockerfile]
FROM openjdk:17-jdk-slim

WORKDIR /app

COPY target/hiv-clinic-backend-*.jar app.jar

EXPOSE 8080

ENTRYPOINT ["java", "-jar", "app.jar"]
\end{lstlisting}

\subsubsection{Frontend Dockerfile}

\begin{lstlisting}[language=Dockerfile, caption=Frontend Dockerfile]
FROM node:18-alpine as build

WORKDIR /app

COPY package*.json ./
RUN npm install

COPY . .
RUN npm run build

FROM nginx:alpine

COPY --from=build /app/dist /usr/share/nginx/html

EXPOSE 80

CMD ["nginx", "-g", "daemon off;"]
\end{lstlisting}

\subsubsection{Docker Compose}

\begin{lstlisting}[language=YAML, caption=Docker Compose]
version: '3.8'

services:
  backend:
    build: .
    ports:
      - "8080:8080"
    environment:
      - SPRING_DATASOURCE_URL=jdbc:sqlserver://db:1433;databaseName=hiv_clinic
      - SPRING_DATASOURCE_USERNAME=sa
      - SPRING_DATASOURCE_PASSWORD=YourStrong@Passw0rd
    depends_on:
      - db

  frontend:
    build: ./frontend
    ports:
      - "80:80"
    depends_on:
      - backend

  db:
    image: mcr.microsoft.com/mssql/server:2019-latest
    environment:
      - ACCEPT_EULA=Y
      - SA_PASSWORD=YourStrong@Passw0rd
    ports:
      - "1433:1433"
    volumes:
      - sqlserver_data:/var/opt/mssql

volumes:
  sqlserver_data:
\end{lstlisting}

\section{Testing Guide}

\subsection{Unit Testing}

\subsubsection{Backend Testing}

\begin{lstlisting}[language=bash, caption=Backend Tests]
# Run all tests
mvn test

# Run tests with coverage
mvn test jacoco:report

# Run specific test class
mvn test -Dtest=UserServiceTest
\end{lstlisting}

\subsubsection{Frontend Testing}

\begin{lstlisting}[language=bash, caption=Frontend Tests]
# Run all tests
npm test

# Run tests with coverage
npm run test:coverage

# Run tests in watch mode
npm run test:watch
\end{lstlisting}

\subsection{Integration Testing}

\subsubsection{API Testing}

Use the provided test scripts:

\begin{lstlisting}[language=bash, caption=API Testing]
# Test database connection
./test-database.sh

# Test notification system
node test-notification-debug.js

# Test simplified notifications
node test-simplified-notifications.js
\end{lstlisting}

\subsection{Performance Testing}

\subsubsection{Load Testing}

\begin{lstlisting}[language=bash, caption=Load Testing]
# Install Apache Bench
sudo apt-get install apache2-utils

# Test API endpoints
ab -n 1000 -c 10 http://localhost:8080/api/health
ab -n 1000 -c 10 -H "Authorization: Bearer <token>" http://localhost:8080/api/appointments/patient/my-appointments
\end{lstlisting}

\section{Maintenance and Support}

\subsection{Regular Maintenance}

\subsubsection{Database Maintenance}

\begin{lstlisting}[language=SQL, caption=Database Maintenance]
-- Regular backup
BACKUP DATABASE hiv_clinic TO DISK = 'C:\backups\hiv_clinic_$(Get-Date -Format 'yyyy-MM-dd').bak'

-- Index maintenance
ALTER INDEX ALL ON Users REBUILD
ALTER INDEX ALL ON Appointments REBUILD
ALTER INDEX ALL ON ARVTreatments REBUILD

-- Statistics update
UPDATE STATISTICS Users
UPDATE STATISTICS Appointments
UPDATE STATISTICS ARVTreatments
\end{lstlisting}

\subsubsection{Log Management}

\begin{lstlisting}[language=bash, caption=Log Management]
# Rotate application logs
logrotate /etc/logrotate.d/hiv-clinic

# Archive old logs
find /var/log/hiv-clinic -name "*.log" -mtime +30 -exec gzip {} \;

# Monitor log size
du -sh /var/log/hiv-clinic/
\end{lstlisting}

\subsection{Monitoring and Alerts}

\subsubsection{Health Checks}

\begin{lstlisting}[language=bash, caption=Health Check Script]
#!/bin/bash

# Check application health
curl -f http://localhost:8080/api/health || echo "Application is down"

# Check database connectivity
sqlcmd -S localhost -d hiv_clinic -Q "SELECT 1" || echo "Database is down"

# Check disk space
df -h | awk '$5 > "80%" {print "Disk space low: " $0}'
\end{lstlisting}

\subsubsection{Performance Monitoring}

\begin{lstlisting}[language=bash, caption=Performance Monitoring]
# Monitor application performance
top -p $(pgrep -f hiv-clinic-backend)

# Monitor database performance
sqlcmd -S localhost -d hiv_clinic -Q "
SELECT 
    DB_NAME(database_id) as DatabaseName,
    COUNT(*) as NumberOfConnections
FROM sys.dm_exec_connections
GROUP BY database_id
"
\end{lstlisting}

\section{Troubleshooting}

\subsection{Common Issues}

\subsubsection{Database Connection Issues}

\textbf{Symptom}: Application fails to start with database connection errors

\textbf{Solutions}:
\begin{enumerate}
    \item Verify SQL Server is running: `systemctl status mssql-server`
    \item Check connection string in application.properties
    \item Verify database exists: `sqlcmd -S localhost -Q "SELECT name FROM sys.databases"`
    \item Check user permissions: `sqlcmd -S localhost -Q "SELECT IS_SRVROLEMEMBER('sysadmin')"`
\end{enumerate}

\subsubsection{Authentication Issues}

\textbf{Symptom}: Users cannot login or receive authentication errors

\textbf{Solutions}:
\begin{enumerate}
    \item Check JWT secret configuration
    \item Verify token expiration settings
    \item Clear browser cache and cookies
    \item Check user account status in database
\end{enumerate}

\subsubsection{Performance Issues}

\textbf{Symptom}: Slow response times or timeouts

\textbf{Solutions}:
\begin{enumerate}
    \item Check database query performance
    \item Monitor application memory usage
    \item Review connection pool settings
    \item Check network connectivity
\end{enumerate}

\subsection{Error Logs}

\subsubsection{Backend Logs}

\begin{lstlisting}[language=bash, caption=Backend Log Monitoring]
# View application logs
tail -f /var/log/hiv-clinic/application.log

# Search for errors
grep -i error /var/log/hiv-clinic/application.log

# Monitor recent activity
tail -100 /var/log/hiv-clinic/application.log | grep $(date +%Y-%m-%d)
\end{lstlisting}

\subsubsection{Database Logs}

\begin{lstlisting}[language=SQL, caption=Database Log Monitoring]
-- Check recent errors
SELECT 
    LogDate,
    ProcessInfo,
    Text
FROM sys.fn_get_audit_file('C:\Program Files\Microsoft SQL Server\MSSQL15.MSSQLSERVER\MSSQL\Log\*.sqlaudit', NULL, NULL)
WHERE LogDate > DATEADD(hour, -1, GETDATE())
ORDER BY LogDate DESC
\end{lstlisting}

\section{Security Considerations}

\subsection{Data Protection}

\begin{itemize}
    \item \textbf{Encryption}: All sensitive data is encrypted at rest and in transit
    \item \textbf{Access Control}: Role-based access control for all system functions
    \item \textbf{Audit Logging}: Comprehensive audit trails for all data access
    \item \textbf{Input Validation}: Server-side validation to prevent injection attacks
    \item \textbf{Session Management}: Secure session handling with timeout
\end{itemize}

\subsection{Security Best Practices}

\begin{enumerate}
    \item \textbf{Regular Updates}: Keep all dependencies updated
    \item \textbf{Strong Passwords}: Enforce strong password policies
    \item \textbf{Network Security}: Use HTTPS in production
    \item \textbf{Backup Security}: Encrypt database backups
    \item \textbf{Monitoring}: Monitor for suspicious activity
\end{enumerate}

\section{Performance Optimization}

\subsection{Database Optimization}

\begin{lstlisting}[language=SQL, caption=Database Indexes]
-- Create indexes for performance
CREATE INDEX IX_Users_Username ON Users(Username)
CREATE INDEX IX_Users_Email ON Users(Email)
CREATE INDEX IX_Appointments_DateTime ON Appointments(AppointmentDateTime)
CREATE INDEX IX_Appointments_Patient ON Appointments(PatientUserID)
CREATE INDEX IX_Appointments_Doctor ON Appointments(DoctorUserID)
CREATE INDEX IX_ARVTreatments_Patient ON ARVTreatments(PatientUserID)
CREATE INDEX IX_Notifications_User ON Notifications(UserID)
\end{lstlisting}

\subsection{Application Optimization}

\begin{itemize}
    \item \textbf{Caching}: Implement Redis for session and data caching
    \item \textbf{Connection Pooling}: Optimize database connection pool settings
    \item \textbf{Query Optimization}: Use efficient database queries
    \item \textbf{Code Splitting}: Implement frontend code splitting
    \item \textbf{Image Optimization}: Compress and optimize images
\end{itemize}

\section{Conclusion}

The HIV Clinic Management System is a comprehensive, secure, and scalable solution for managing HIV clinic operations. The system provides all necessary features for patient care, appointment management, treatment tracking, and administrative functions.

Key achievements of this release:

\begin{itemize}
    \item \textbf{Complete Feature Set}: All planned features implemented and tested
    \item \textbf{Security Compliance}: Healthcare-grade security and privacy protection
    \item \textbf{Performance Optimization}: Optimized for production use
    \item \textbf{Comprehensive Documentation}: Complete user and technical documentation
    \item \textbf{Deployment Ready}: Production-ready deployment procedures
\end{itemize}

The system is now ready for production deployment and can support the daily operations of HIV clinics while ensuring the highest standards of patient care and data security.

\end{document}